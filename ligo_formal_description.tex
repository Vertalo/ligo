\documentclass[10pt,a4paper]{article}
\usepackage[utf8]{inputenc}
\usepackage[english]{babel}
\usepackage[fleqn]{amsmath}
\usepackage{amsfonts}
\usepackage{amssymb}
\usepackage{graphicx}
\usepackage{listings}
\usepackage[left=2cm,right=2cm,top=2cm,bottom=2cm]{geometry}
\title{Ligo Formal Description}
\lstset{language=caml}
\author{}
\date{}
\begin{document}
\maketitle
\section*{ Syntax }
The following describe the syntax of the simplify AST which is an internal of LIGO. The concrete syntax will be different depending of the choosen one but all the caterogies are present and the corresponding evaluation are the same

A LIGO program is a succession of declarations and expressions. Declarations add bindings to the environment while expressions are evaluated and yield values

\begin{align*}
variables\ (x)
\end{align*}
\begin{align*}
label\ (l)
\end{align*}
\begin{align*}
constructor\ (c)
\end{align*}
\begin{align*}
declaration\ (d)\ = 
\\
 \ &|\ type\ x\ is\ te                     \tag{Type\ declaration}
\\
 \ &|\ const\ x\ (:\ te)?\ =\ e               \tag{Constant\ declaration}
\\
 \ &|\ var  \ x\ (:\ te)?\ =\ e               \tag{Variable\ declaration}
\end{align*}
\begin{align*}
expression\ (e)\ = 
\\
 \ &|\ value                            \tag{values}
\\
 \ &|\ built_in                         \tag{built-in\ function}
\\
 \ &|\ x                                \tag{variables}
\\
 \ &|\ \lambda x\ .\ expr                        \tag{lambda}
\\
 \ &|\ e1\ e2                            \tag{application}
\\
 \ &|\ let\ x\ =\ e1\ in\ e2                 \tag{let\ in}
\\
 \ &|\ (\ e_i\ )                          \tag{tuple}
\\
 \ &|\ c\ e                              \tag{constructor}
\\
 \ &| \{\ l_i\ =\ e_i \}                    \tag{record}
\\
 \ &|\ [\ e1_i\ =\ e2_i\ ]                  \tag{map}
\\
 \ &|\ [[\ e1_i\ =\ e2_i\ ]]                \tag{big\ map}
\\
 \ &|\ [\ e_i\ ]                          \tag{list}
\\
 \ &| \{\ e_i \}                          \tag{set}
\\
 \ &|\ e(.a_i)                          \tag{accessor}
\\
 \ &|\ e1[e2]                           \tag{look\ up}
\\
 \ &|\ match\ e\ with\ matching            \tag{matching}
\\
 \ &|\ e1;\ e2                           \tag{sequence}
\\
 \ &|\ while\ e1\ do\ e2                   \tag{loop}
\\
 \ &|\ x(.a_i)\ =\ e                      \tag{assign}
\\
 \ &|\ SKIP                             \tag{skip}
\\
 \ &|\ e\ as\ T                           \tag{ascription}
\end{align*}
\begin{align*}
type\ expression\ (te)\ =
\\
 \ &|\ te\ (*\ te_i)+                     \tag{type\ of\ tuple}
\\
 \ &|\ (|\ l_i\ of\ te_i)                  \tag{type\ of\ sum}
\\
 \ &| \{\ l_i\ :\ te_i \}                   \tag{type\ of\ record}
\\
 \ &|\ te1\ \rightarrow \ te2                       \tag{type\ of\ function}
\\
 \ &|\ l                                \tag{type\ of\ variable}
\\
 \ &|\ l\ (te_i)                         \tag{type\ of\ built\ in\ function}
\end{align*}
\begin{align*}
value\ (v)\ = 
\\
 \ &|\ literal                          \tag{values\ of\ built-in\ types}
\\
 \ &|\ c\ v                              \tag{values\ of\ construct\ types}
\\
 \ &|\ \lambda x\ .\ expr                        \tag{lambda}
\end{align*}
\begin{align*}
literal\ =
\\
 \ &|\ unit                             \tag{}
\\
 \ &|\ bool                             \tag{}
\\
 \ &|\ int                              \tag{}
\\
 \ &|\ nat                              \tag{}
\\
 \ &|\ mutez                            \tag{}
\\
 \ &|\ string                           \tag{}
\\
 \ &|\ bytes                            \tag{}
\\
 \ &|\ address                          \tag{}
\\
 \ &|\ timestamp                        \tag{}
\\
 \ &|\ operation                        \tag{}
\end{align*}
\begin{align*}
access\ (a)\ = 
\\
 \ &|\ int                              \tag{for\ tuples}
\\
 \ &|\ string                           \tag{for\ record}
\\
 \ &|\ e                                \tag{for\ map}
\end{align*}
\begin{align*}
matching\ (m)\ = 
\\
 \ &| \{\ true\ =>\ e;\ false\ =>\ e;\}        \tag{match\ bool}
\\
 \ &| \{\ nil \ =>\ e;\ cons(hd::tl)\ =>\ e;\} \tag{match\ list}
\\
 \ &| \{\ none\ =>\ e;\ some(x) \ =>\ e;\}     \tag{match\ option}
\\
 \ &|\ (x_i) \ =>\ e                      \tag{match\ tuple}
\\
 \ &|\ (const_i(x_i)\ =>\ e_i\ )           \tag{match\ variant}
\end{align*}
\begin{align*}
matching\ value\ (mv)\ = 
\\
 \ &| \{\ true\ =>\ v;\ false\ =>\ v;\}        \tag{match\ bool\ value}
\\
 \ &| \{\ nil \ =>\ v;\ cons(hd::tl)\ =>\ v;\} \tag{match\ list\ value}
\\
 \ &| \{\ none\ =>\ v;\ some(x) \ =>\ v;\}     \tag{match\ option\ value}
\\
 \ &|\ (x_i) \ =>\ v                      \tag{match\ tuple\ value}
\\
 \ &|\ (const_i(x_i)\ =>\ v_i\ )           \tag{match\ variant\ value}
\end{align*}
\newpage
\section*{ Evaluation of expression }
The following describe how expression are evaluated to yield expressions 

\subsection*{ base }
%Values are not evaluted
\begin{equation}x\ \rightarrow \ v\ (corresponding\ value\ in\ the\ environment)\ \tag{\ E-VARIABLE\ }\end{equation}
\begin{equation}built\ in\ (e_i)\ \rightarrow \ built\ in\ result                           \ (*\ evaluated\ depending\ on\ each\ case\ *)\ \tag{\ E-BUILTIN\ }\end{equation}
\begin{equation}(\lambda x.e)\ v\ \rightarrow \ [\ x\ \rightarrow \ v\ ]\ e\ \tag{\ E-LAMBDA\ }\end{equation}
\begin{equation}\frac{e1\ \rightarrow \ e1'}{e1\ e2\ \rightarrow \ e1'\ e2}\ \tag{\ E-APP1\ }\end{equation}
\begin{equation}\frac{e2\ \rightarrow \ e2'}{v1\ e2\ \rightarrow \ v1\ e2'}\ \tag{\ E-APP2\ }\end{equation}
\begin{equation}\frac{e1\ \rightarrow \ e1'}{let\ x=e1\ in\ e2\ \rightarrow \ let\ x=e1'\ in\ e2}\ \tag{\ E-LET\ }\end{equation}
\begin{equation}let\ x=v1\ in\ e2\ \rightarrow \ [x\rightarrow v1]\ e2\ \tag{\ E-LETIN\ }\end{equation} 
\begin{equation}\frac{e1\ \rightarrow \ e1'}{e1;\ e2\ \rightarrow \ e1';\ e2}\ \tag{\ E-SEQ\ }\end{equation}
\begin{equation}unit;\ e2\ \rightarrow \ e2\ \tag{\ E-SEQNEXT\ }\end{equation}
\begin{equation}\frac{e1\ \rightarrow \ e1'}{while\ e1\ then\ e2\ \rightarrow \ while\ e1'\ then\ e2}\ \tag{\ E-LOOP\ }\end{equation}
\begin{equation}while\ true(=e1)\ then\ e2\ \rightarrow \ e2;\ while\ e1\ then\ e2\ \tag{\ E-LOOPTRUE\ }\end{equation}
\begin{equation}while\ false\ then\ e2\ \rightarrow \ unit\ \tag{\ E-LOOPFALSE\ }\end{equation}
\begin{equation}SKIP\ \rightarrow \ unit\ \tag{\ E-SKIP\ }\end{equation}
\begin{equation}\frac{e\ \rightarrow \ e'}{e\ as\ T\ \rightarrow \ e'\ as\ T}\ \tag{\ E-ASCR1\ }\end{equation}
\begin{equation}v\ as\ T \ \rightarrow \ v\ \tag{\ E-ASCR2\ }\end{equation}  

\subsection*{ data structure }
\begin{equation}\frac{e\ \rightarrow \ e'}{c\ e\ \rightarrow \ c\ e'}\ \tag{\ E-CONST\ }\end{equation}
\begin{equation}\frac{e_j\ \rightarrow \ e_j'}{(v_i,\ e_j,\ e_k)\ \rightarrow \ (v_i,\ e_j',\ e_k)}\ \tag{\ E-TUPLES\ }\end{equation}
\begin{equation}\frac{e_j\ \rightarrow \ e_j'}{\{l_i=v_i,\ l_j=e_j,\ l_k=e_k\}\ \rightarrow \ \{l_i=v_i,\ l_j=e_j',\ l_k=e_k\}}\ \tag{\ E-RECORDS\ }\end{equation}
\begin{equation}\frac{e2_j\ \rightarrow \ e2_j'}{[e1_i=v_i,\ e1_j=e2_j,\ e1_k=e2_k]\ \rightarrow \ [e1_i=v_i,\ e1_j=e2_j',\ e1_k=e2_k]}\ \tag{\ E-MAP\ }\end{equation}
\begin{equation}\frac{e2_j\ \rightarrow \ e2_j'}{[[e1_i=v_i,\ e1_j=e2_j,\ e1_k=e2_k]]\ \rightarrow \ [[e1_i=v_i,\ e1_j=e2_j',\ e1_k=e2_k]]}\ \tag{\ E-BIGMAP\ }\end{equation}
\begin{equation}\frac{e_j\ \rightarrow \ e_j'}{[v_i,\ e_j,\ e_k]\ \rightarrow \ [v_i,\ e_j',\ e_k]}\ \tag{\ E-LIST\ }\end{equation}
\begin{equation}\frac{e_j\ \rightarrow \ e_j'}{\{v_i,\ e_j,\ e_k\}\ \rightarrow \ \{v_i,\ e_j',\ e_k\}}\ \tag{\ E-SET\ }\end{equation}
\begin{equation}\frac{e\ \rightarrow \ e'}{e(.a_i)\ \rightarrow \ e'(.a_i)}\ \tag{\ E-ACCESS\ }\end{equation}

\subsection*{ look up }
\begin{equation}(v_i)[j]     \ \rightarrow \ v_j\ \tag{\ E-LUPTUPLE\ }\end{equation} 
\begin{equation}\{l_i=v_i\}[lj] \ \rightarrow \ v_j\ \tag{\ E-LUPRECORD\ }\end{equation}
\begin{equation}[e_i=v_i][ej] \ \rightarrow \ v_j\ \tag{\ E-LUPMAP\ }\end{equation}
\begin{equation}[[e_i=v_i]][ej] \ \rightarrow \ v_j\ \tag{\ E-LUPBIGMAP\ }\end{equation}
\begin{equation}[v_i][j]     \ \rightarrow \ v_j\ \tag{\ E-LUPLIST\ }\end{equation}
\begin{equation}\{v_i\}[j]     \ \rightarrow \ v_j\ \tag{\ E-LUPSET\ }\end{equation}
\begin{equation}\frac{e\ \rightarrow \ e'}{x(.a_i)\ =\ e\ \rightarrow \ x(.a_i)\ =\ e'}\ \tag{\ E-ASSIGN\ }\end{equation}
\begin{equation}x(.a_i)\ =\ v\ \rightarrow \ x'(.a_i)\ with\ x'\ as\ x\ with\ field\ (.a_i)\ replace\ by\ v\ \tag{\ E-ASSIGN2\ }\end{equation}

\subsection*{ matching }
\begin{equation}\frac{e\ \rightarrow \ e'}{match\ e\ with\ m\ \rightarrow \ match\ e'\ with\ m}\ \tag{\ E-MATCH1\ }\end{equation}
\begin{equation}\frac{m\ \rightarrow \ m'}{match\ v\ with\ m\ \rightarrow \ match\ v \ with\ m'}\ \tag{\ E-MATCH2\ }\end{equation}
\begin{equation}match\ v\ with\ mv\ \rightarrow \ v'\ if\ v\ =>\ v'\ in\ mv\ \tag{\ E-MATCH \ }\end{equation}
\begin{equation}\frac{e1\ \rightarrow \ e1'}{\{\ true\ =>\ e1;\ false\ =>\ e2;\}\ \rightarrow \ \{\ true\ =>\ e1';\ false\ =>\ e2;\}}\ \tag{\ E-MAcTHBOOL1\ }\end{equation}
\begin{equation}\frac{e2\ \rightarrow \ e2'}{\{\ true\ =>\ v1;\ false\ =>\ e2;\}\ \rightarrow \ \{\ true\ =>\ v1;\ false\ =>\ e2';\}}\ \tag{\ E-MAcTHBOOL2\ }\end{equation}
\begin{equation}\frac{e1\ \rightarrow \ e1'}{\{\ nil \ =>\ e1;\ cons(hd::tl)\ =>\ e2;\}\ \rightarrow \ \{\ nil\ =>\ e1';\ cons(hd::tl)\ =>\ e2;\}}\ \tag{\ E-MATCHLIST1\ }\end{equation}
\begin{equation}\frac{e2\ \rightarrow \ e2'}{\{\ nil \ =>\ v1;\ cons(hd::tl)\ =>\ e2;\}\ \rightarrow \ \{\ nil\ =>\ v1;\ cons(hd::tl)\ =>\ e2';\}}\ \tag{\ E-MATCHLIST2\ }\end{equation}
\begin{equation}\frac{e1\ \rightarrow \ e1'}{\{\ none\ =>\ e1;\ some(x) \ =>\ e2;\}\ \rightarrow \ \{\ none\ =>\ e1';\ some(x)\ =>\ e2;\ \}}\ \tag{\ E-MATCHOPT1\ }\end{equation}
\begin{equation}\frac{e2\ \rightarrow \ e2'}{\{\ none\ =>\ v1;\ some(x) \ =>\ e2;\}\ \rightarrow \ \{\ none\ =>\ v1';\ some(x)\ =>\ e2';\ \}}\ \tag{\ E-MATCHOPT2\ }\end{equation}
\begin{equation}\frac{e\ \rightarrow \ e'}{(x_i) \ =>\ e\ \rightarrow \ (x_i)\ =>\ e'}\ \tag{\ E-MATCHTUPLE\ }\end{equation}            
\begin{equation}\frac{e_j\ \rightarrow \ e_j'}{(c_i(x_i)\ =>\ v_i,\ c_j(x_j)\ =>\ e_j,\ c_k(x_k)\ =>\ e_k)\ \rightarrow \ (c_i(x_i)\ =>\ v_i,\ c_j(x_j)=>e_j',\ c_k(x_k)=>e_k)}\ \tag{\ E-MATCHVARIANT\ }\end{equation}    



\section*{ Derive form }
The following describe equivalent notation. Meaning one could be drop for the AST without change the CSTs
\begin{equation*}e1;\ e2\ \iff \ (\lambda x:Unit.e1)\ e2\ with\ x\ not\ a\ free\ variable\ in\ e1 \ \end{equation*}
\begin{equation*}let\ x=e1\ in\ e2\ \iff\ (\lambda x:T1.e2)\ e1 \ \end{equation*}
\end{document}
